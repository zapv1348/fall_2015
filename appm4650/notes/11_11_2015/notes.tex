\documentclass{article}

\usepackage{graphicx}
\usepackage{subcaption}
\usepackage{amsmath}
\usepackage{amsfonts}

\begin{document}

\author{Zachary Vogel}
\date{\today}
\title{Notes in APPM 4650\\Adam Norris}

\maketitle

\section{Project Continued}
Cone of energy...\\
\[\frac{d\theta}{d\sigma}=\delta e^{\theta}-\theta\]
$\delta<\frac{1}{e}$ fizzle.\\
$\delta>\frac{1}{e}$ explosion\\
$\delta=\frac{1}{5}$ and $\delta=1$.\\
for fizzle solutions your just going to integrate the ode with RK4. This should asymptote to $\theta_f$.\\
Late solution for fizzle will be when $\theta_f=\delta e^{\theta_f}$.\\
Early solution for fizzle will be from $\frac{d\theta}{d\sigma}=\delta+(\delta-1)\theta$.\\
\[\frac{d\theta}{\delta+(\delta-1)\theta}=d\sigma\]
\[\frac{d\theta}{\theta+(\frac{\delta}{\delta-1})}=(\delta-1)d\sigma\]
\[\theta=\frac{\delta}{\delta-1}(e^{(\delta-1)\sigma}-1)\]
This early solution works for both fizzle and explosion.\\
for explosion, solving for $\sigma$ is uesful.\\
\[\sigma=\left (\frac{1}{\delta-1}\right)\ln\left [\cfrac{\theta+\cfrac{\delta}{\delta-1}}{\cfrac{\delta}{\delta-1}}\right ]\]

Late solution for the explosion:\\
\[\frac{dy}{dx}=\frac{1}{\delta e^{x}-x}\]
$y=\sigma$, $x=\theta$.\\
\[r=\cfrac{\ln\left (\cfrac{\theta+\cfrac{\delta}{\delta-1}}{\cfrac{\delta}{\delta-1}}\right)}{\delta-1}\]
\[\lim_{\delta\to 1}\sigma=\]
should give ou sigma vs theta at the beginning..\\

on explosion:\\
\[\frac{d\theta}{d\sigma}\approx \delta e^{\theta}\]
\[e^{-\theta}d\theta=\delta d\sigma\]
\[-e^{-\theta}=\sigma\delta+c\]
\[\theta\bigg|_{\sigma\to\sigma_{expl}}\to\infty\]
\[c=-\delta \sigma_{expl}\]
\[\sigma \delta=\delta\sigma_{expl}-e^{-\theta}\]
\[\sigma=\sigma_{expl}-\frac{1}{\delta}e^{-\theta}\]
late explosion approximation.\\
$\sigma_{expl}$ is unkown though.\\
\[\frac{d\sigma}{d\theta}=\frac{1}{\delta e^{\theta}-\theta}\]
\[\int_{\theta=0}^\infty\frac{d\sigma}{d\theta}d\theta=\int_{\theta=0}^\infty\frac{d\theta}{\delta e^{\theta}-\theta}\]
\[0=\theta_{expl}-\sigma\bigg|_{\theta=0}\]
\[\theta_{expl}=int_{\theta=0}^\infty \frac{d\theta}{\delta e^{\theta}-\theta}\]
Simpson's something for solution.\\
by the time $\theta$ is about 10 your good.\\jacobian


\end{document}
