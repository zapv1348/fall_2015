\documentclass{article}

\usepackage[margin=1in]{geometry}
\usepackage{amsmath}
\usepackage{amsfonts}

\author{Zachary Vogel}
\title{Notes in Dynamics and Manuevering\\ ECEN 5008}
\date{\today}


\begin{document}
\maketitle
\section{Trajectory Exploration}
System=Sliding Car
\[\begin{pmatrix}a_{\text{long}}\\a_{\text{lat}}\end{pmatrix}=\begin{pmatrix}\dot{v}\\v(\omega+\dot{\beta})\end{pmatrix}=R_z(-\beta)\begin{pmatrix}u_1\\F_y(\beta)\end{pmatrix}\]
\[\dot{\omega}=u_2\]

Convient Set of trajectories: constant trajectories (aka Equillibrium Points)\\
Equilibirum Manifold:\\
Sliding Car - constant speed circles\\
parameterize by $(v,a_{\text{lat}})$\\
\[\begin{pmatrix}\dot{v}=0\\a_{\text{lat}}\end{pmatrix}=R_z(-\beta)\begin{pmatrix}u_1\\F_y(\beta)\end{pmatrix}\]
Interesting that this doesn't depend on velocity\\
Equillibrium Manifold can be largely parameterized by $a_{\text{lat}}$.\\
\[\bar{\beta}(a_{\text{lat}}), \bar{u}_1(a_{\text{lat}}\]

by setting the derivatives equal to zero, you can get $\dot{\omega}=u_2$ trivially and then solve the two equations for $\dot{v}$ and $\dot{\beta}$ easily.\\

Like to use continuation (homotopy) start with an easy problem and morph it into a hard problem.\\

quasi-static approach: design things so curvature doesn't change too fast. If they aren't changing to fast, at each instance of time your just going around a constant speed circle.

\end{document}
