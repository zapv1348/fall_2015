\documentclass{article}

\usepackage[margin=1in]{geometry}
\usepackage{amsmath}
\usepackage{amsfonts}
\usepackage{graphicx}


\title{Homework 9: Advanced Linear Systems (ECEN 5448)}
\author{Zachary Vogel}
\date{\today}

\newcommand{\rank}{\text{rank}}

\begin{document}
\maketitle

\section*{Problem 1}
Given $a<b$. Show that the set of polynomials on any interval $(a,b)$ is an independent set in the vector space of functions.\\
In this problem we are effectively trying to show that:
\[\sum_{n=0}^\infty c_n*t^{n}\neq 0\]
if and only if $c_n=0$. Starting with:
\[c_1*1+c_2*t=0\]
If we assume this is true for $c_1,c_2\neq 0$, we can construct a contradiction by taking the first derivative:
\[c_2*t=0\]
which implies $c_2=0$, and $c_1=0$, giving us a contradiction. Next, we move on to the case with 3 elements.
\[c_1*1+c_2*t+c_3*t^2=0\]
By making the same assumption that $c_n\neq 0$ and taking the second derivative we get:
\[2*c_3=0\implies\ c_3=0\implies c_2=0\implies c_1=0\]
Now, we show that if the $c_{n+1}$ term is zero for the $n+1$ element case, the rest of the $c$ elements are zero by taking the $n$th derivative. So, we have:
\[c_11+c_2t+c_3t^2+\dots+c_{n}t^n+c_{n+1}t^{n+1}=0\]
It should be clear that if $c_{n+1}=0$ all other $c$'s are zero. Taking the $n+1$th derivative we get:
\[c_{n+1}(n+1)!=0\]
which implies $c_{n+1}$ is zero, implying all $c$'s are zero.

\section*{Problem 2}
Perform the controllability transformation on the system $\dot{x}=Ax+Bu$ where:
\[A=\begin{pmatrix}1&1&0\\1&1&1\\0&1&1\end{pmatrix}\quad,\ B=\begin{pmatrix}0\\1\\0\end{pmatrix}\]
First, I need $A^2$.
\[A^2=\begin{bmatrix}2&2&1\\2&3&2\\1&2&2\end{bmatrix}\]
Then the controllability Gramian is:
\[C=\begin{bmatrix}0&1&2\\1&1&3\\0&1&2\end{bmatrix}\]
Note the rank of this matrix is 2. Finding the orthonormal basis of C, we get two vectors:
\[v_1=\begin{pmatrix}0.48723\\0.72472\\0.48723\end{pmatrix}\quad v_2=\begin{pmatrix}-0.51246\\0.68904\\-0.51246\end{pmatrix}\]
Now we need another vector $v_3$ orthonormal to our first two to form the T matrix.\\
\[\begin{array}{c}v_3=\begin{pmatrix}-0.70711\\0\\0.70711\end{pmatrix}\\[2em]T=\begin{pmatrix}0.48723&0.72472&0.48723\\-0.51256&0.68904&-0.51256\\-0.70711&0&0.70711\end{pmatrix}\end{array}\]
Now we can do the controllability transformation:
\[\dot{z}=TAT^{-1}z+TBu=\begin{pmatrix}2.4124&-0.071338&0\\-0.071338&-0.41241&0\\0&0&1\end{pmatrix}z+\begin{pmatrix}0.72472\\0.68904\\0\end{pmatrix}u\]


\section*{Problem 3}
Is the system from Problem 2 stabalizable?\\
Stabalizability is proven by showing the matrix:
\[(A-\lambda*I\ B)\]
has full row rank for all $Re(\lambda)\geq 0$. The eigenvalues of the original matrix are $-0.41421$, $1$, and $2.41421$. Therefore,
\[S1=(A-I\ B)=\begin{bmatrix}0&1&0&0\\1&0&1&1\\0&1&0&0\end{bmatrix}\]
must be full rank, but it only has rank 2. Therefore, the system is not stabalizable.


\section*{Problem 4}
For the linear system:
\[\begin{array}{c}\dot{x}=Ax+Bu\\y=Cx+Du\end{array}\]
show that linear state transformation $z=Rx$, preserves the transfer function where R is an invertible matrix.\\
The transfer function of the original system is:
\[Y(s)=C(s*I-A)^{-1}B+D\]
The new system for the transform is:
\[\begin{array}{c}\dot{z}=TAT^{-1}z+TBu\\y=C*T^{-1}z+Du\end{array}\]
Then, the transfer function becomes
\[\frac{Y(s)}{U(s)}=CT^{-1}(sI-TAT^{-1})^{-1}TB+D\]
\[=CT^{-1}(T^{-1}(s*I-TAT^{-1}))^{-1}B+D\]
\[=C(T^{-1}(sI-TAT^{-1})T)^{-1}B+D\]
\[=C(T^{-1}sIT-IAI)^{-1}B+D\]
\[=C(sI-A)^{-1}B+D\]
Completing the proof.

\section*{Problem 5}
We know that for dynamics:
\[\begin{array}{c}\dot{x}=Ax+Bu\\y=Cx+Du\end{array}\]
there exists a matrix R, from the controllability transformation, such that the linear state transformation $z=Rx$ results in dynamics:
\[\begin{array}{c}\begin{pmatrix}\dot{z}_1\\\dot{z}_2\end{pmatrix}=\begin{pmatrix}A_c&A_{cu}\\0&A_u\end{pmatrix}\begin{pmatrix}z_1\\z_2\end{pmatrix}+\begin{pmatrix}B_c\\0\end{pmatrix}u\\y=\begin{pmatrix}C_c&C_u\end{pmatrix}\begin{pmatrix}z_1\\z_2\end{pmatrix}+Du\end{array}\]
where $A_c,B_c$ are a controllable pair. Show that the transfer function of the original system and the transfer function of the reduced system:
\[\begin{array}{c}\dot{z}_1=A_cz_1+B_cu\\y=C_cz_1+Du\end{array}\]
are the same. As a result, if a system is not controllable, one can reduce the dimension of the internal state.\\
\[\frac{Y(s)}{U(s)}=(C_c\ C_u)\left(sI-\begin{pmatrix}A_c&A_{cu}\\0&A_u\end{pmatrix}\right)^{-1}\begin{pmatrix}B_c\\0\end{pmatrix}+D\]
Evaluating $(sI-A)^{-1}$:
\[(sI-A)^{-1}=\begin{pmatrix}sI-A_c&-A_{cu}\\0&sI-A_u\end{pmatrix}^{-1}=\begin{pmatrix}(sI-A_{c})^{-1}&-(sI-A_c)^{-1}A_{cu}A_u^{-1}\\0&A_{u}^{-1}\end{pmatrix}\]
Then, plugging into the transfer function we get:
\[\frac{Y(s)}{U(s)}=C_c(sI-A_c)^{-1}B_c+D\]
Then, due to the last problem, we know this is equivalent to the transfer function for the original system.\\


\end{document}
