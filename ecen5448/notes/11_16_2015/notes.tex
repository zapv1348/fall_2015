\documentclass{article}

\usepackage[margin=1in]{geometry}
\usepackage{graphicx}
\usepackage{amsmath}
\usepackage{amsfonts}


\author{Zahary Vogel}
\date{\today}
\title{Notes in ECEN 5448}

\newcommand{\rank}{\text{rank}}
\newcommand{\rng}{\text{range}}

\begin{document}
\maketitle


\section{Controllability}
\[\dot{x}=Ax+Bu\]
System is controllable:
\begin{enumerate}
    \item iff $e^{-A\tau}B$ is linearly independent
    \item iff $W(t,0)=(\int_0^Te^{-A\tau}BB^Te^{-A^T\tau}d\tau)$ is full rank.\\
    \item iff $\rank(B AB A^2B \dots A^{n-1}B)=n$
\end{enumerate}
to drive the system from $x(0)\to 0$ $\forall x(0)$.\\

We say that a state $x\in\mathbb{R}^n$ is reachable if $\exists T,\exists u$ such that:\\
\[x(T)=\int_0^Te^{A(T-\tau)}Bu(\tau)d\tau=\int_0^Te^{As}*B*(u(T-s)=g(s))dx\]
System is reachable if for any state in $\mathbb{R}^n$ we can drive the system from 0 to that point.\\
A system is reachable iff rows of: $e^{A\tau}B$ are linearly independent iff:\\
\[W_R(0,T)=-\int_0^Te^{A\tau}BB^Te^{A^T\tau}d\tau\]
is invertible, iff $\rank(B AB A^2B \dots A^{n-1}B)=n$\\
AKA a system is reachable iff it is controllable. All states that are reachable are controllable.\\

\section{Controllability of LTV Systems}
Supposedly, all this analysis except the rank condition will work for time varying systems.\\
Baby version of time varying systems:\\
Consider the system:
\[\dot{x}=B(t)u(t)\]
For a given initial condition, $x(t_0)\in\mathbb{R}^n$, what states $x(t_1)$ are reachable at $t_1$?\\
Suppose that you can make $x(t_1)$ with $u$\\
\[x(t_1)=\int_{t_0}^{t_1}B(t)u(t)dt+x(t_0)\]
\[\implies x(t_1)-x(t_0)\in\rng\left (\int_{t_0}^{t_1}B(t)dt\right )=\mathcal{R}(B(t))\]
The claim is that $\mathcal{R}(B(t))=\rng\left (\int_{t_0}^{t_1}B(t)B^T(t)dt\right)$\\
How to show this?\\
Suppose that \\
\[y\in\rng\left (\int_{t_0}^{t_1}B(t)B^T(t)dt\right )\implies \exists \tilde{u}\in\mathbb{R}^m\]
\[y=\int_{t_0}^{t_1}B(t)B^T(t)dt\tilde{u}\]
Note that $y\in\mathcal{R}(B(t))$ as $u(t)=B^T(t)\tilde{u}$ would give $\int_{t_0}^{t_1}B(t)B^T(t)\tilde{u}dt=y$.\\

Suppose that $y\in \mathcal{R}(B(t))$ and $y\not{\in}\rng\left (\int_{t_0}^{t_1}B(t)B^T(t)dt\right )$.\\

Let $V\in\mathbb{R}^n$ be a linear subspace and $y\not{\in}V$. Then, if $y\neq 0$ there exists a $u\in\mathbb{R}^n$, s.t. $u^Tv=0 \forall v\in V$ and $u^Ty\neq 0$.\\
Proof: Let B be an orthonormal basis for V and C be an orthonormal basis for $\mathbb{R}^n$(?). Furthermore, $y^Tc_i\neq 0$ for some $c_i\in\mathbb{C}$(?).\\
Note that $c_i\bot v,\forall v\in V$.\\
$V^{\bot}=\{u\bigg| u^Tv=0 \forall u\in V\}$.\\
\[\implies \exists c\in\mathbb{R}^n, c^Ty\neq 0\]
but $c^T\int_{t_0}^{t_1}B(t)B^T(t)dt\neq 0$.\\
\[0=c^T\int_{t_0}^{t_1}B(t)B^T(t)dt\]
(note: B must be continous) iff\\
\[c^t\int_{t_0}^{t_1}B(t)B^T(t)dt c=0=\int_{t_0}^{t_1}\lvert\lvert B^T(t)c\rvert\rvert^2dt\]
so this holds iff\\
\[B^T(t)c=0\]
But $y\in\rng(B(t))$ iff $\exists u(t)$\\
\[y=\int_{t_0}^{t_1}B(t)u(t)dt\implies 0\neq c^Ty=\int_{t_0}^{t_1}c^TB(t)u(t)dt=0\]
thus we have a contradiction.\\
Also note that the matrix forming the range of $B(t)$ is extremely dependent on $t_0,t_1$.\\

Now for the general system:\\
\[\dot{x}(t)=A(t)x(t)+B(t)u(t)\]
$x(t_1)\in\mathbb{R}^n$ at time $t_1$ is reachable from $x(t_0)$ at $t_0$ if $\exists u(t)$ such that:\\
\[x(t_1)=\Phi(t_1,t_0)x(t_0)+\int_{t_0}^{t_1}\Phi(t_1,t))Bu(t)dt\]
Theorem: $x(t_1)$ at $t_1$ is reachable from $x(t_0)$ at $t_0$ iff:\\
\[\Phi(t_0,t_1)x(t_1)-x(t_0)\in\rng\left (\int_{t_0}^{t_1}\Phi(t_0,t)B(t)B^T(t)\Phi^T(t_0,t)dt\right )\]
Reachability Gramian:\\
\[W_R(t_0,t_1)=\int_{t_0}^{t_1}\Phi(t_0,t_1)B(t)B^T(t)\Phi^T(t_0,t)dt\]

\section{Controllability Decomp}
Fact: Suppose that $\rank(B AB \dots A^{n-1}B)=R<n$. Then, $\exists$ invertible $T$ such that if we let $z=Tx$, then:\\
\[\dot{z}=\begin{pmatrix}\bar{A}_{11} & \bar{A}_{12}\\0 & \bar{A}_{22}\end{pmatrix}z+\begin{pmatrix}\bar{B}\\0\end{pmatrix}u\]
Proof: Let 
\[T=\begin{pmatrix}v_1^H\\v_2^H\\\dots\\v_r^H\\w_1^H\\\dots\\w_{n-1}^H\end{pmatrix}\]
such that $(v_1,\dots,v_r)$ is an orthonormal basis for $\rng(B\dots A^{n-1}B)$ and $(w_1,\dots,w_{n-1})$ is an orthonormal basis for $R^{\bot}$.\\
This will result in the controllability decomp.\\
Showing this in homework is bonus.\\

\section{Fourth check of Controllability}
PBH-eigenvector test: Popov-Belevitch-Hautus\\
any P in controls is Popov, any K is kalman.\\
The pair $(A,B)$ is controllable iff $\rank(A-\lambda I; B)=n$ $\forall \lambda\in\mathbb{R}$\\
Proof Sketch: Case that A is diagnolizable. Suppose that $wA=\Lambda w$ where:
\[w=\begin{pmatrix}w_1^H\\\dots\\w_n^H\end{pmatrix}\]
and\\
\[\Lambda=\text{diag}(\lambda_1,\lambda_2,\dots,\lambda_n\]
If we let $z=wx$ then:\\
\[\dot{z}=w\dot{x}=wAw^Hz+wBu\]
Note that $\rank(A-\lambda I,B)<n$ iff there exists $w$ such that $w^T(A-\lambda I)=0$ and $w^TB=0$ which would mean $W^T$ is an eigenvector for $A^T$.\\
So a left eigenvector for A must be orthogonal to B.\\
$\implies$ if $\rank()<n\implies \exists w_1,\dots,w_n \bot B$ for some $r\geq 1\ \implies$\\
\[\dot{z}=\Lambda z+\begin{pmatrix}\bar{B}\\0\end{pmatrix}u\]
and this system is clearly not controllable.\\

It can be shown (?) that linear transformation $z=Tx$ using an invertible $T$ does not effect the controllability.\\

What are the various controllability tests useful for?\\
Suppose that we have $\dot{x}=Ax+Bu$. A is not stable.\\
Question: $\exists$ $m\times n$ matrix K; $u=Kx$ such that the system is stable.\\
i.e. $\dot{x}=Ax+BKx=(A+BK)x$ is stable?\\

This is called Stabilizability.\\
The difference between this and controllability is that controllability is finite time horizon.\\


\end{document}
