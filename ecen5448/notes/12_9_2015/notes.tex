\documentclass{article}

\usepackage[margin=1in]{geometry}
\usepackage{graphicx}
\usepackage{amsmath}
\usepackage{amsfonts}


\author{Zahary Vogel}
\date{\today}
\title{Notes in ECEN 5448}

\begin{document}
\maketitle


\section*{Fundamental Theorem of Lin Alg Proof}
Very non-trivial.\\
row rank of $A_{m\times n}$=column rank A.\\
Let $v_1,\dots,v_k$ be the independent rows of A that span the rowspace of $A$.\\
\[A=\begin{bmatrix}A_1\\\vdots\\A_m\end{bmatrix}\]
row rank(A)=k\\
Let $u_i\in \mathbb{R}^n=Av_i^T$ for $i=1,\dots,k$.\\
We show that $u_1,\dots,u_k$ are independent.\\
Suppose that this isn't true, and prove by contradiction. $\implies \exists \alpha_1,\dots,\alpha_k$:\\
\[\alpha_1u_1+\dots+\alpha_ku_k=0\]
\[\leftrightarrow \alpha_1Av_1^T+\dots+\alpha_kAv_k^T=0\]
this implies that:
\[A(\alpha_1v_1^T+\dots+\alpha_kv_k^T)=0\]
\[z=\alpha_1v_1^T+\dots+\alpha_kv_k^T\]
this means that z is orthogonal to $A_i$ including $v_1,\dots,v_k$. IN particular, z is orthogonal to z, which implies $\lvert\lvert z\rvert\rvert^2=0\implies z=0$. Since, $v_1,\dots,v_k$ are linearly independent $\alpha_1=\dots=\alpha_k=0$ which means $u_i$s are linearly indepedent.\\
So we have found k indepedent vectors that are in the column space that are independent. Therefore, the column rank is greater than or equal to the row rank.\\
Repeat this argument for $u_i*A=v_i$, or use the same argument for A transpose. Then you get the other inequality.\\

\section*{Controllability Comment}
want to move from one point in space to another.\\
$\exists u(t)$:
\[-x(0)=\int_0^Te^{-A\tau}Bu(\tau)d\tau\]
\[y=\int_0^Te^{-A\tau}Bu(\tau)d\tau \text{for some control} u(\tau)\]
is equal to the range of the controllability matrix. So just find a vector that isn't in the range of the controllability matrix and then find a vector that isn't in that.\\

\section*{Homework 10}
\subsection*{Problem 1}
\[\dot{x}=Ax+Bu\]
\[A=\begin{pmatrix}0&1\\1&0\end{pmatrix}\]
\[B=\begin{pmatrix}1\\1\end{pmatrix}\]
\subsubsection*{(a)}
Use the rank test to conclude that $(A,B)$ is not controllable.\\
\[C=\begin{pmatrix}1&1\\1&1\end{pmatrix}\]
It's rank is clearly 1.\\
\subsubsection*{(b)}
Use the PBH test to conclude that $(A,B)$ is stabilizable.
\[\text{rank}(A-\lambda I B)=n\]
for all $\lambda$ with $Re(\lambda)\geq 0$\\
$\lambda_1=1$, $\lambda_2=-1$.\\
\[\begin{pmatrix}-1&1&1\\1&-1&1\end{pmatrix}\]
Note that: 
\[det(\begin{pmatrix}-\lambda & 1\\1&1\end{pmatrix}=-\lambda-1\neq 0\]
for $\lambda$ with Re$(\lambda)\geq 0$
\subsubsection*{(c)}
Find a T such that the transformation $z=Tx$ takes the system to triangular Controller Form.\\
Want T such that $TAT^{-1}=\begin{pmatrix}A_c&A_{cu}\\0&A_u\end{pmatrix}$ and $TB=\begin{pmatrix}B_c\\0\end{pmatrix}$.
\[T=\frac{1}{\sqrt{2}}\begin{pmatrix}1&1\\1&-1\end{pmatrix},\quad T^{-1}=\frac{1}{\sqrt{2}}\begin{pmatrix}1&1\\1&-1\end{pmatrix}\]
\[TAT^{-1}=\frac{1}{2}\begin{pmatrix}2&0\\0&-2\end{pmatrix}\]
\[TB=\frac{1}{\sqrt{2}}\begin{pmatrix}2\\0\end{pmatrix}\]
\subsubsection*{(d)}
Find a stabilizing feedback in the transformed coordinates $u=-Gz$ for $z=Tx$.\\
\[\dot{z}=\begin{pmatrix}1&0\\0&-1\end{pmatrix}z+\begin{pmatrix}\sqrt{2}\\0\end{pmatrix}u(t)\]
\[u(t)=-\begin{pmatrix}-\sqrt{2}&0\end{pmatrix}z\]
Thus, the system is:
\[\dot{z}=\begin{pmatrix}-1&0\\0&-1\end{pmatrix}z\]
\subsubsection*{(e)}
Now express this in the original coordinates $u=-Kx$.\\
\[u=(-\sqrt{2}\quad 0)z=(-\sqrt{2}\quad 0)Tx\]
\[=(-sqrt{2}\quad 0)\frac{1}{\sqrt{2}}\begin{pmatrix}1&1\\1&-1\end{pmatrix}x\]
\[=(-1\quad -1)x\]
checking this:
\[Ax+Bu=\begin{pmatrix}0&1\\1&0\end{pmatrix}x+\begin{pmatrix}1\\1\end{pmatrix}(-1\quad -1)x\]
\[=\begin{pmatrix}-1 & 0\\0&-1\end{pmatrix}x\]
which is Hurwitz.

\subsection*{Problem 2}
Show that for any $A$, there exists a $\mu>0$ such that:
\[-\mu I-A\]
is a stability matrix.\\
\[C_{A\mu}(\lambda)=\det(A_\mu-\lambda I)=\det (-A-\mu I-\lambda I)=\det(-A-(\mu+\lambda)I)=c_{-A}(\mu+\lambda)\]
So if $\lambda_0$ is an eigenvalue of $-A-\mu I$ then, $C_{-A}(\mu+\lambda_0)=0$ i.e. $\mu+\lambda_0$ is an eigenvalue for $-A$. Thus, for large enough $\mu$, $\lambda_0<0$.\\
Let $(\lambda_1,\dots,\lambda_n)$ be eigenvalues of $-A$.\\
If $\mu>\lvert Re(\lambda_1)\rvert,\dots,\lvert Re(\lambda_n)\rvert$ then $\mu+\lambda_0=\lambda_i$ for some $\lambda_0=\lambda_i-\mu$\\
\[\implies Re(\lambda_0)=Re(\lambda_i)-\mu<0\]
because $\mu$ is bigger than the real part of $\lambda_1$.\\

\subsection*{Problem 3}
Use the PBH test to show that if $(A,B)$ is a controllable pair, then $(-\mu I-A, B)$ is a controllable pair.\\
$(A,B)$ controllable $\leftrightarrow$ $(-A-\mu I, B)$ is controllable.
\[\text{rank}(A-\lambda I\quad B)<n\]
iff $c^T(A-\lambda I)=0\leftrightarrow c^TA=\lambda c^T$ $c^TB=0$\\
Left eigenvectors of $A=$ left eigenvectors of $(-A-\mu I)$\\
\[c^T(-A-\mu I)=-c^TA-\mu c^T=-\lambda c^T-\mu c^T=(-\lambda-\mu)c^T\]

\subsection*{Problem 4}
Consider the inverted pendulum model:
\[\begin{pmatrix}\dot{x_1}\\\dot{x_2}\end{pmatrix}=\begin{pmatrix}x_2\\\sin(x_1)-(x_2+x_2\lvert x_2\rvert)\end{pmatrix}+\begin{pmatrix}0\\1\end{pmatrix}u(t)\]
where $u(t)$ is the input torque. Linearize th dynamics around the equilibrium point $(0,0)$. Let $\dot{x}=Ax+bu$ be that linearization. For the linearized model, derive the equations for the minimum energy controller that derives the dynamics from $(0.2,0)$ to close neighborhood of $(0,0)$ in 1 second. In other words, derive the Riccati equation and the coresponding controller for the following optimal control problem:
\[\min_{u_{[0,1]}}\left (\int_0^1 u^2(\tau)d\tau+10\lvert\lvert x(1)-(0,0)\rvert\rvert^2\right )\]
subject to: $\dot{x}=Ax+bu$.\\

\[\dot{x}=\begin{pmatrix}0&1\\1&-1\end{pmatrix}x+\begin{pmatrix}0\\1\end{pmatrix}u(t)\]
\[Q(t)=0,\quad R(t)=1\]
\[M=10I\]
\[\min \int_0^! u^T Rud\tau+x^T(1)Mx(1)\]
\[\dot{P}=PA+A^TP-PBR^{-1}B^TD,\quad P(1)=M\]
\[=P\begin{pmatrix}0 & 1\\1 & -1\end{pmatrix}+\begin{pmatrix}0&1\\1&-1\end{pmatrix}P+P\begin{pmatrix}0&0\\0&1\end{pmatrix}P\]
we get:
\[V^O(x,t)=x^TP(t)x\]
The optimizer would be the state feedback:
\[u^0(t)=-K(t)x(t)=-R^{-1}B^TP(t)x(t)=\begin{pmatrix}0\\-1\end{pmatrix}P(t)x(t)\]

\end{document}
